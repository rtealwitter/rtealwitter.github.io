\documentclass[11pt]{article}
% 11 pt font required for Computer Modern (latex default)
% 1 inch margins minimum 
\usepackage[tmargin=1in,bmargin=1in,lmargin=1in,rmargin=1in]{geometry}
\setlength{\parskip}{0.7em}
\usepackage[usenames,dvipsnames,svgnames,table]{xcolor}
\PassOptionsToPackage{hyphens}{url}\usepackage[colorlinks=true,linkcolor=black,urlcolor=Black,citecolor=MidnightBlue]{hyperref}

% Put in shortcut for name of school that I'm applying to
\newcommand{\school}{ETH Zurich ITS}
% Shortcut for name of undergraduate research program
\newcommand{\program}{Davidson Research Initiative}
% grep -lR --null 'documentclass' . | xargs -0 latexmk -cd

\usepackage{amsmath,amsfonts,amsthm,amssymb}
\usepackage{enumitem}

\usepackage{titlesec}
\titleformat*{\section}{\large\bfseries}

\usepackage{wrapfig}
\usepackage[font=small]{caption}
\usepackage{subcaption}
\usepackage{float}
\usepackage{graphicx}
\usepackage{comment}   
\usepackage{cleveref}
\usepackage{mathtools}
\usepackage{csquotes}

\usepackage{times}

\begin{document}

\begin{center}
	\Large \textbf{Diversity Statement} \\
	\vspace{.25em}
	\large{R. {\color{teal}Teal} Witter}
\end{center}

Computer science faces a disturbing lack of diversity.
Women earn only 26\% of bachelor's degrees in mathematics and computer science while Black students earn only 8\% \cite{national2023diversity}.
I had heard the statistics but only \textit{understood} the enormity of the problem when a close friend took her first coding class.
Because of undiagnosed medical issues, she struggled even to see the screen where she was learning to code.
While the professor offered many contact hours, my friend felt intimidated in the crowded, white- and male-dominated office hours.
Struggling with health issues and without prior computer science experience, she failed the class.
The professor, expecting her to approach him, never asked why she was performing poorly and gave a failing grade.
Believing she might not be able to graduate, my friend grew depressed and even contemplated suicide.

How could a professor I knew as caring allow an intelligent and dedicated student to suffer so needlessly?
I believe there are two underlying problems.
First, computer science can be intimidating and exclusive, discouraging under-represented students from participating.
Second, access to computer science education is inequitable, excluding under-served students from educational opportunities.
In my teaching, research, and advising, I work to address these underlying problems.

%\noindent {\large\textbf{Teaching: Proactive Support}}

I proactively support the students in my classes.
In order to understand how they're doing, I ask students to fill out a check-in form after every lecture.
The form assesses which concepts are confusing and invites students to pose the questions they do not feel comfortable asking in class.
I use the responses to revisit the material at the beginning of the next lecture, answering the questions from the check-in forms and clarifying points of confusion.
I also use the responses to identify students who are struggling as soon as their understanding lapses.
For these students, I gently invite them to my office and tutor them on the concepts that they find challenging.
When a systemic issue emerges, I research college resources and direct students to the relevant support.
By immediately identifying where they struggle, I intervene long before students would have even received a poor grade on an assignment.

%\noindent {\large\textbf{Teaching: Inclusive Learning}}

I build inclusive learning environments into my classes.
In lectures, I show students that learning is a process and mistakes are normal, highlighting the errors I inadvertently make and what I learn from each one.
In class activities, I give students the opportunity to actively learn on low-stakes problems in assigned groups.
I make it clear that it is more important that every group member understands what the group has accomplished than that one group members knows the solution.
Students that are comfortable with the material explain what they understand while their peers ask questions about what they do not yet grasp.
After several classes in one group, I re-assign groups so that students who normally explain their reasoning can practice learning from each other and students who normally receive help can practice leading the group discussion.
The low-stakes problems have components of varying difficulty:
On the easier components, every student experiences what it's like to solve a problem.
On the more difficult components, every student experiences what it's like to approach a problem that cannot be easily solved.
I design the group activities so that every student learns from their peers and teaches them, solves problems and wrestles with difficult questions.
The message is that everyone belongs in computer science; every question and point of confusion is as welcome and normal as every explanation and insight.

%\noindent {\large\textbf{Research}}

I work on research projects that support equitable access to education.
Reach Out and Read Colorado (RORCO) is an early childhood literacy nonprofit that encourages under-served families to read to their children.
When I approached them to offer my academic expertise, they asked for a way to evaluate the effectiveness of their work without a (resource-intensive) randomized control study.
In the language of causal inference, RORCO needed an estimate of treatment effect from a natural experiment where the RORCO `treatment' was assigned in a non-randomized way.
Using RORCO's internal records and publicly accessible education information, I compiled a dataset with dozens of features on literacy outcomes for hundreds of thousands of students spanning nearly a decade.
To analyze the data, I designed an estimator specifically for their natural experiment setting with provable performance guarantees even when treatments are assigned in a non-randomized way.
Consistent with the early childhood literacy literature, my findings indicate that RORCO has a positive impact on literacy outcomes for the under-served students they support.
Already, RORCO has leveraged these results to acquire additional funding, expanding their program and the support they offer to under-served families.

%\noindent {\large\textbf{Volunteering}}

I volunteer with under-served students to share my love of computer science. 
When I started my graduate studies, I began volunteering at a local high school for English Language Learners.
I offered weekly coding practice as an extracurricular activity to a dedicated and enthusiastic group of students.
We used Python code to investigate questions the students asked like: 
``When is it worth taking out loans for a college education?''
``How do transmissible illnesses like COVID-19 spread?''
``When are electric vehicles more economically viable than gas vehicles?''
The students learned how to isolate the components of a question, outline a high level solution, and dive in with technical know-how to develop an answer.
In the several years I worked with the students, I watched their analytical and coding skills grow along with their confidence.
I was overjoyed that every student who participated in the extracurricular enrolled in college after graduation and decided to pursue computer science.
Once the cohort I worked with graduated, I turned to advising.

%\noindent {\large\textbf{advising}}

I advise students who might not otherwise have the chance to participate in academic research.
Instead of choosing students that reflect the typical demographics of computer science departments, I work with students that more closely reflect the general population.
The projects I advise are carefully designed as a gentle introduction to research.
We begin with a high level overview of a research topic that the student finds interesting.
As we explore the topic, I direct the students to resources and, in regular meetings, address their questions.
The resources are paired with exercises to give practice working in the research topic.
When we review their work together, I emphasize their insights and encourage their progress.
When the student feels ready, we formulate a specific research question for them to investigate.
The question is a step beyond the exercises and gives a taste of the research experience in a supportive environment.
Through advising, I work to make research access more equitable.

%\noindent {\large\textbf{Going Forward}}

Computer science can be intimidating and access is often inequitable.
As a professor, I will extend and broaden my work to address these issues.
In my classes, I will provide proactive support and build inclusive communities, ensuring that all students learn in a welcoming and supportive environment.
In my research, I will develop long-term collaborations with nonprofit organizations, ensuring the research my students and I work on has positive social impact.
In my advising roles, I will support \textit{all} students, sharing computer science and academic research especially with students who would otherwise not have access.

\bibliographystyle{alpha}
\bibliography{references}

\end{document}
