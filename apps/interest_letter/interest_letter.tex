\documentclass[11pt]{article}
% 11 pt font required for Computer Modern (latex default)
% 1 inch margins minimum 
\usepackage[tmargin=1in,bmargin=1in,lmargin=1in,rmargin=1in]{geometry}
\setlength{\parskip}{0.7em}
\usepackage[usenames,dvipsnames,svgnames,table]{xcolor}
\PassOptionsToPackage{hyphens}{url}\usepackage[colorlinks=true,linkcolor=black,urlcolor=Black,citecolor=MidnightBlue]{hyperref}

% Put in shortcut for name of school that I'm applying to
\newcommand{\school}{ETH Zurich ITS}
% Shortcut for name of undergraduate research program
\newcommand{\program}{Davidson Research Initiative}
% grep -lR --null 'documentclass' . | xargs -0 latexmk -cd

\usepackage{amsmath,amsfonts,amsthm,amssymb}
\usepackage{enumitem}

\usepackage{titlesec}
\titleformat*{\section}{\large\bfseries}

\usepackage{wrapfig}
\usepackage[font=small]{caption}
\usepackage{subcaption}
\usepackage{float}
\usepackage{graphicx}
\usepackage{comment}   
\usepackage{cleveref}
\usepackage{mathtools}
\usepackage{csquotes}

\usepackage{times}
\usepackage{fancyhdr}
\pagestyle{fancy}
\lhead{\includegraphics[width=4cm]{../tandon_long_color.eps}}
\chead{\Large \textbf{Interest in the \school}}
\rhead{\large \href{https://www.rtealwitter.com/}{R. {\color{teal}Teal} Witter}}
\cfoot{}

\begin{document}

{\setlength{\parindent}{0cm}

{\large \textbf{Interdisciplinary Research}}

As a theoretical computer scientist with a focus on algorithms for social good, my research naturally aligns with the Santa Fe Institute's interdisciplinary approach to complex systems. My work on explainable AI, fairness in machine learning, and dynamic decision-making systems has the potential to contribute significantly to SFI's research themes, particularly in the areas of complexity, computation, and society.

My research on improving explainable AI methods, such as Shapley and Banzhaf value estimators, can provide valuable tools for researchers across disciplines at SFI. These methods can help demystify complex AI systems, making them more transparent and interpretable for researchers studying social systems, economic models, or ecological networks. By bridging the gap between advanced AI techniques and their practical applications in various fields, my work would foster new collaborations and insights at the intersection of computer science and other disciplines represented at SFI.

My work on online decision-making in dynamic environments, such as the NYC Open Streets Project and resource allocation for restless bandits, aligns well with SFI's focus on complex adaptive systems. These projects demonstrate how computational approaches can be applied to real-world problems involving multiple stakeholders, changing environments, and competing objectives. At SFI, I would expand this research to incorporate insights from fields like ecology, economics, and social sciences, potentially leading to more robust and adaptable algorithms for societal challenges.

The unique environment at SFI, with its emphasis on interdisciplinary research, would significantly enhance my work. Engaging with researchers from diverse backgrounds would provide new perspectives and methodologies to tackle complex societal problems. For instance:
(1) Collaborating with social scientists would refine my approaches to fairness in machine learning, incorporating deeper insights into human behavior and social dynamics.
(2) Working alongside ecologists and biologists would enhance my research on resource allocation for endangered species reintroduction, leading to more effective conservation strategies.
(3) Interacting with economists and game theorists would provide new frameworks for analyzing and optimizing dynamic decision-making processes in urban planning and nonprofit resource allocation.
%
%\begin{itemize}[leftmargin=*]
%    \item Collaborating with social scientists would refine my approaches to fairness in machine learning, incorporating deeper insights into human behavior and social dynamics.
%    \item Working alongside ecologists and biologists would enhance my research on resource allocation for endangered species reintroduction, leading to more effective conservation strategies.
%    \item Interacting with economists and game theorists would provide new frameworks for analyzing and optimizing dynamic decision-making processes in urban planning and nonprofit resource allocation.
%\end{itemize}

\textbf{Advantages of SFI's Unique Environment}

Several aspects of my current and future research would greatly benefit from SFI's non-traditional academic environment:

\begin{enumerate}[leftmargin=*]
%    \item \textbf{Interdisciplinary Focus}: My work on algorithms for social good requires deep engagement with practitioners and stakeholders across various fields. SFI's emphasis on breaking down disciplinary barriers would provide unparalleled opportunities for such collaborations.
    \item \textbf{Flexibility in Research Directions}: The freedom to explore unconventional ideas and methodologies at SFI would be invaluable for my research. For instance, my interest in distortion-free watermarking for responsible AI would evolve in unexpected directions through interactions with researchers studying information theory, cryptography, or even biological signaling systems.
    \item \textbf{Complex Systems Perspective}: SFI's focus on complex systems would provide a unique lens through which to view and expand my work on dynamic decision-making and societal polarization. This perspective would lead to novel insights and methodologies that might be difficult to develop in a traditional computer science department.
\end{enumerate}

In conclusion, the Santa Fe Institute's unique interdisciplinary environment and focus on complex systems would provide an ideal setting for expanding my research on algorithms for social good. In turn, my expertise in theoretical computer science and its applications to societal challenges would contribute to SFI's mission of addressing complex problems through collaborative, boundary-crossing scientific inquiry. The potential for growth and impact makes this opportunity extremely exciting.
}

\end{document}