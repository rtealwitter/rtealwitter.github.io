\documentclass[11pt]{article}
% 11 pt font required for Computer Modern (latex default)
% 1 inch margins minimum 
\usepackage[tmargin=1in,bmargin=1in,lmargin=1in,rmargin=1in]{geometry}
\setlength{\parskip}{0.7em}
\usepackage[usenames,dvipsnames,svgnames,table]{xcolor}
\PassOptionsToPackage{hyphens}{url}\usepackage[colorlinks=true,linkcolor=black,urlcolor=Black,citecolor=MidnightBlue]{hyperref}

% Put in shortcut for name of school that I'm applying to
\newcommand{\school}{ETH Zurich ITS}
% Shortcut for name of undergraduate research program
\newcommand{\program}{Davidson Research Initiative}
% grep -lR --null 'documentclass' . | xargs -0 latexmk -cd

\usepackage{amsmath,amsfonts,amsthm,amssymb}
\usepackage{enumitem}

\usepackage{titlesec}
\titleformat*{\section}{\large\bfseries}

\usepackage{wrapfig}
\usepackage[font=small]{caption}
\usepackage{subcaption}
\usepackage{float}
\usepackage{graphicx}
\usepackage{comment}   
\usepackage{cleveref}
\usepackage{mathtools}
\usepackage{csquotes}

\usepackage{times}

\usepackage{fancyhdr}
\pagestyle{fancy}
\lhead{\includegraphics[width=4cm]{../tandon_long_color.eps}}
\chead{\Large \textbf{Teaching Statement}}
\rhead{\large \href{https://www.rtealwitter.com/}{R. {\color{teal}Teal} Witter}}
\cfoot{}

\begin{document}

{\setlength{\parindent}{0cm}

I love teaching computer science, and a major component of my interest in \school~is its commitment to excellent undergraduate education. While PhD students are not required to teach at NYU, I have sought out teaching and mentorship opportunities. I served as a teaching assistant for eight courses at NYU, compiled a complete set of notes for one course, developed and taught two winter term courses at Middlebury College, founded a math modeling club at a local high school for English language learners, and advised four undergraduate and high school student research projects. My teaching and mentoring is rooted in fostering deep understanding, critical thinking, and real-world applications of computer science concepts.

I teach students how to derive knowledge rather than memorize facts. In my first Deep Learning course\footnote{\url{https://www.rtealwitter.com/deeplearning2023/}}, I introduced a three-part framework—model, loss function, and optimizer—for understanding complex neural networks. For example, when learning about convolutional networks, students dissected each component’s role in efficient image processing. By the end of the course, students analyzed cutting-edge concepts like diffusion models, confidently breaking them down into familiar components. I teach in this way not only to deepen student understanding but also to cultivate critical thinking skills, equipping students to tackle novel challenges and apply deep learning in their careers.

In order to effectively tailor my teaching, I regularly check in with students through assigned questionnaires. When covering neural network architectures, I might ask students to explain the role of activation functions in their own words. The fluency and depth of their responses reveal their level of understanding, enabling me to identify areas needing reinforcement. I begin the next class with a targeted recap, using visualizations and step-by-step breakdowns to clarify the concept. For example, in a lecture on gradient descent in my deep learning class, I asked about the importance of tuning step size. Based on a common confusion in the answers, I began the next class by drawing concrete functions where large step sizes risk missing local optima while small step sizes can slow down optimization. My adaptive teaching resonates with students. As one student remarked,

\begin{displayquote}
	\textit{I really liked how the professor would break the lecture into a recap, logistics, and new material. It made it a lot easier to stay engaged and give us time to fully understand material from the previous day, without jumping straight into new material.}
\end{displayquote}

I leverage the questionnaire to mitigate barriers to learning, particularly those faced by students from underrepresented groups or with unique challenges. By reaching out to students who appear confused on the questionnaire, I've uncovered issues ranging from sleep apnea to financial struggles, and directed students to appropriate resources—all before the first assignment was due and without students having to overcome potential cultural or social barriers to seeking help. By addressing challenges early, I create a more level playing field where all students, regardless of their background or circumstances, have the opportunity to excel. After seeing students who initially struggled succeed in my class, I shared the positive results of the method with Gustavo Sandoval and the questionnaire is now being used in his graduate deep learning course at NYU.

While designing my second course on Randomized Algorithms for Data Science\footnote{\url{https://www.rtealwitter.com/rads2024/}}, I focused on the ``why'' behind concepts, bridging the gap between abstract mathematical ideas and real-world applications. In the course, I guided students through the process of applying complex mathematical concepts to solve tangible problems. When introducing hashing, I explained its theoretical foundations, then demonstrated how it's used for efficient load balancing on servers. Likewise, I connected cosine similarity and dimensionality reduction to music recognition services like Shazam, helping students grasp how these abstract concepts power everyday technology. When exploring eigenvectors, I motivated them in the context of social network analysis, showing how this mathematical tool can reveal insights about community structures. By framing abstract material in terms of its practical applications, I not only deepen students' understanding of the material but also cultivate their ability to see the broader implications of the algorithms they're learning, preparing them to innovate in their future careers.

%Similarly, I connect material to student experiences and teach skills to approach complicated problems. When I started graduate school, I sought a way to volunteer in my local community. As a coach for the M3 Challenge at a high school for English language learners, I offered weekly coding practice as an extracurricular activity. We used Python to investigate real-world questions posed by the students, such as the economic viability of college loans, the spread of COVID-19, and the cost-effectiveness of electric vehicles. Because of the language barrier and the fact that the club was an extracurricular, I learned to adapt my material on the fly to student interest. However, over several years, I watched students become fluent at coding and confident breaking down complicated topics. Every student in the group enrolled in college and, at least initially, majored in computer science.

I foster an inclusive learning environment that actively counters the “weed-out” culture often associated with computer science. In my Winter 2024 Randomized Algorithms for Data Science course, I introduced practices that encouraged a culture of support, where mistakes were normalized as an integral part of the learning process. I adapted my teaching style to emphasize the mistakes I inadvertently make in lectures, emphasizing how each one deepens my understanding. I also incorporated group activities where students participate as both learners and teachers, tackling a mix of simple and complex problems to gain familiarity with problem-solving. By frequently rotating group assignments, I ensure that every student has the opportunity to lead discussions and contribute as team members. This approach not only reinforces that everyone belongs in computer science but also emphasizes the value of collaborative learning. In the words of one student,

\begin{displayquote}
	\textit{I really enjoyed the group activities as it allowed me to learn from other people just as much as it was a test of my knowledge of the topic. The notion of working with other people is something that I feel has gone understated in Computer Science, but you learn so much more when you do.}
\end{displayquote}

I design my own research to make space for student collaboration: My research agenda is multi-faceted so students can carve out projects that interest them, gaining new skills while applying what they already know. In my research advising, I work to get students “unstuck”: I focus on the challenges students describe, diving into code to collaboratively debug or drawing on a whiteboard to jointly gain new insights. I have advised four undergraduate and high school students on research projects and I am excited to continue advising through \school’s programs like the \program.

My comprehensive and empathetic teaching style, coupled with my broad research experience, has prepared me to effectively teach a wide range of courses, including foundational courses like Algorithms, Machine Learning, Deep Learning, Data Science, Computational Complexity, and more specialized courses like Explainable AI, Causal Inference, Fairness in Machine Learning, Quantum Algorithms, Reinforcement Learning, Analysis of Boolean Functions, and Social Networks.

My teaching and mentoring is rooted in fostering deep understanding, critical thinking, and real-world application of computer science concepts. By emphasizing the ``why'' behind ideas, creating inclusive learning environments, and connecting abstract theories to practical applications, I strive to inspire and equip students to solve real problems. In the words of one student,

\begin{displayquote}
	\textit{Teal [...] was empathetic and understanding. I always felt like I could approach him with any question without judgement. I could tell that he cared about my learning and encouraged me to learn as much as possible and to think beyond grades. He also handled his own mistakes carefully and was always open to criticism. It is obvious that Teal is an amazing teacher that cares about his students' learning and experience.}
\end{displayquote}

%After creating inclusive spaces where students know that I care, I share my excitement for computer science and teach students problem-solving skills that will serve them beyond the classroom.
}

\end{document}