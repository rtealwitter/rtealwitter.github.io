\documentclass[11pt]{article}
% 11 pt font required for Computer Modern (latex default)
% 1 inch margins minimum 
\usepackage[tmargin=1in,bmargin=1in,lmargin=1in,rmargin=1in]{geometry}
\setlength{\parskip}{0.7em}
\usepackage[usenames,dvipsnames,svgnames,table]{xcolor}
\PassOptionsToPackage{hyphens}{url}\usepackage[colorlinks=true,linkcolor=black,urlcolor=Black,citecolor=MidnightBlue]{hyperref}

% Put in shortcut for name of school that I'm applying to
\newcommand{\school}{ETH Zurich ITS}
% Shortcut for name of undergraduate research program
\newcommand{\program}{Davidson Research Initiative}
% grep -lR --null 'documentclass' . | xargs -0 latexmk -cd

\usepackage{amsmath,amsfonts,amsthm,amssymb}
\usepackage{enumitem}

\usepackage{titlesec}
\titleformat*{\section}{\large\bfseries}

\usepackage{wrapfig}
\usepackage[font=small]{caption}
\usepackage{subcaption}
\usepackage{float}
\usepackage{graphicx}
\usepackage{comment}   
\usepackage{cleveref}
\usepackage{mathtools}
\usepackage{csquotes}

\usepackage{times}


% Signature fonts
\usepackage{frcursive,wedn,wela,pbsi}
\usepackage[T1]{fontenc}

\usepackage{fancyhdr}
\pagestyle{fancy}
\lhead{\includegraphics[width=4cm]{../tandon_long_color.eps}}
\chead{\Large \textbf{Cover Letter}}
\rhead{\large \href{https://www.rtealwitter.com/}{R. {\color{teal}Teal} Witter}}
\cfoot{}

\begin{document}

{\setlength{\parindent}{0cm}

Dear Search Committee,

I am writing to apply for the Assistant Professor position in the Math \& Computer Science Department, as advertised on the \school~website. I am a PhD candidate in the Computer Science and Engineering Department at New York University, where I expect to graduate in May 2025. I believe my work at the intersection of theoretical computer science, machine learning, and data science would complement and extend \school’s current strengths in data analytics, machine learning, and human-computer interaction. Further, I believe my experience and interest in inclusive teaching would be an excellent fit at \school.

I am a theoretical computer scientist studying algorithms for social good. As computing becomes ubiquitous, algorithms should be effective (e.g., solve the problems they are designed to) while operating in fair and transparent ways (e.g., people can trust and understand the decisions they make). To make machine learning decisions more interpretable, I’ve developed algorithms with theoretical guarantees—Leverage SHAP and Kernel Banzhaf—which offer robust and efficient solutions for explainable AI. I’ve also applied these techniques in collaboration with nonprofits, developing algorithms for treatment effect estimation that are simple yet reliable for high-stakes applications.

In addition to making algorithms more transparent and trustworthy, my research enhances the effectiveness of algorithms in some of the most compelling social good applications: systems where algorithms interact with and adapt to changing environments. I’ve designed reinforcement learning models for urban planning projects like the NYC Open Streets Project, optimizing for both congestion reduction and safety. My work on resource allocation, particularly in the context of restless multi-armed bandits, has laid theoretical groundwork for addressing complex optimization problems like wildlife conservation. Throughout my research, I maintain a balance between theoretical depth and practical impact, often collaborating across disciplines to ensure my algorithms address genuine societal needs.

My multi-faceted research agenda is designed to engage students with diverse interests and skills. I've mentored four undergraduate and high school students, tailoring projects to challenge students while building on existing knowledge. For instance, I recently guided a Barnard College undergraduate in exploring Gaussian Splatting for 3D image reconstruction, which combined her interests with cutting-edge computer vision techniques. This project not only advanced her skills in data processing and algorithm implementation but also connected to my interest in interpretable models. I look forward to mentoring students at \school~on projects that advance algorithms for social good, particularly through programs like the \program.

I love teaching computer science, and a major factor in my interest in \school~is its commitment to excellent undergraduate education. While PhD students are not required to teach at NYU, I have sought out teaching and mentorship opportunities. I served as a teaching assistant for eight courses at NYU, compiled a complete set of notes for one course, developed and taught two winter term courses at Middlebury College, and created a math modeling club at a local high school for English language learners.

I adapt my teaching to actively support students. In my Winter 2023 Deep Learning course\footnote{\url{https://www.rtealwitter.com/deeplearning2023/}}, I started assigning a check-in form after every class. This approach allows me to gauge student understanding and identify confusion without the potential stigma of asking questions in class or approaching me directly. For example, in a lecture on gradient descent, I asked about the importance of tuning step size. Based on a common confusion in the answers, I began the next class by drawing concrete examples where large step sizes risk missing local optima while small step sizes can slow down optimization. My approach to adaptive teaching not only gives students a chance to process the material but also allows me to continuously improve my explanations. By proactively addressing areas of difficulty, I ensure that all students, regardless of their background and learning style, have equitable access to clarification and support.

I foster an inclusive learning environment that actively counters the “weed-out” culture often associated with computer science. In my Winter 2024 Randomized Algorithms for Data Science course\footnote{\url{https://www.rtealwitter.com/rads2024/}}, I introduced practices that encouraged a culture of support, where mistakes were normalized as an integral part of the learning process. During lectures, I highlighted the mistakes that I inadvertently made and what I learned from each one. During in-class activities, I designed group problem-solving sessions so every student practiced explaining an answer and trying a new approach, struggling with a question and successfully solving a problem. My goal was to emphasize that everyone belongs in computer science; every question and point of confusion is as welcome and normal as every explanation and insight.

I am excited to bring my research expertise and dedication to inclusive education to \school, where I believe I can contribute significantly to both student learning and the vibrant academic community.

Sincerely,

\vspace{1em}

%{\cursive R. Teal Witter}
%{\wedn R. Teal Witter}
%{\Large \textbf{\wela R. Teal Witter}}
{\bsifamily R. Teal Witter}

}

\end{document}