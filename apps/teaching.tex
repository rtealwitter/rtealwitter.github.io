\documentclass[11pt]{article}
% 11 pt font required for Computer Modern (latex default)
% 1 inch margins minimum 
\usepackage[tmargin=1in,bmargin=1in,lmargin=1in,rmargin=1in]{geometry}
\setlength{\parskip}{0.7em}
\usepackage[usenames,dvipsnames,svgnames,table]{xcolor}
\PassOptionsToPackage{hyphens}{url}\usepackage[colorlinks=true,linkcolor=black,urlcolor=Black,citecolor=MidnightBlue]{hyperref}

% Put in shortcut for name of school that I'm applying to
\newcommand{\school}{ETH Zurich ITS}
% Shortcut for name of undergraduate research program
\newcommand{\program}{Davidson Research Initiative}
% grep -lR --null 'documentclass' . | xargs -0 latexmk -cd

\usepackage{amsmath,amsfonts,amsthm,amssymb}
\usepackage{enumitem}

\usepackage{titlesec}
\titleformat*{\section}{\large\bfseries}

\usepackage{wrapfig}
\usepackage[font=small]{caption}
\usepackage{subcaption}
\usepackage{float}
\usepackage{graphicx}
\usepackage{comment}   
\usepackage{cleveref}
\usepackage{mathtools}
\usepackage{csquotes}

\usepackage{times}

\begin{document}

\begin{center}
	\Large \textbf{Teaching Statement} \\
	\vspace{.25em}
	\large{R. {\color{teal}Teal} Witter}
\end{center}

I love teaching because of the challenge of answering an insightful question, the deeper understanding from a new perspective, and the moment of comprehension in students' eyes.
As a teacher, I carefully structure both the content and style of my classes to support every student's learning.
In recognition that teaching is a skill that will take a lifetime to hone, I reflect on each class and the feedback I receive to make the next one even better. 

In terms of content, I structure each class to develop transferrable problem-solving skills.
As a student, I often wondered why professors asked us to solve particular problems and whether I would use (or even remember) specific skills in the future.
As a teacher, I ensure each course is built around a meta-cognitive approach to problem-solving that transcends specific applications.
Students see the problem-solving framework applied in lectures, explore it in collaborative in-class activities, and practice applying it on homework problems.
At the end of every course, students walk out with a broad problem-solving toolkit that transfers to the problems they will face in their careers.

In terms of teaching style, I structure each class to support student learning.
During my first course, I concentrated on connecting with students; developing a check-in form to tailor my teaching to students' questions and provide proactive support before any student fell behind.
During my second course, I concentrated on student engagement; developing in-class group activities to practice problem-solving and class resources to support all learning styles.

%\noindent {\large\textbf{Deep Learning}}

In Winter 2023, I taught an undergraduate course on deep learning.\footnote{\url{https://www.rtealwitter.com/deeplearning2023/}}
The framework of the course is a three-step recipe for deep learning: a model to generate predictions, a loss function to measure the quality of the predictions, and an optimizer to improve the model by minimizing the loss.
In the course, we use the framework to unify seemingly disparate topics:
We view convolutional networks as an efficient model, contrastive learning as an unsupervised loss, and diffusion as a model and loss pair.
Armed with the framework, students can understand deep learning algorithms deployed at scale and the latest academic research.

During the deep learning course, I developed a check-in form to engage directly with students' questions and proactively support students' needs.
By regularly checking in with students, I tailor my teaching to students' understanding.
After every class, I ask each student to fill out a check-in form to gauge their understanding.
With their questions and points of confusion in mind, I prepare a review at the beginning of the next class to answer their questions and clarify the prior material.
In the words of one student:
\begin{displayquote}
\textit{
I really liked how the professor would break the lecture into a recap, logistics, and new material. It made it a lot easier to stay engaged and give us time to fully understand material from the previous day, without jumping straight into new material.
}
\end{displayquote}

Instead of waiting for students to come to me, I use the check-in form to offer proactive support.
I read each check-in form response after every class.
If it seems like a student is confused or overwhelmed, I immediately contact them to offer one-on-one office hours and, if applicable, direction to campus resources.
The check-in form enables me to address student needs even if they are not comfortable approaching me to voice them.
After sharing my positive experience, the check-in form is now being used in the graduate deep learning course at NYU.

%\noindent {\large\textbf{Randomized Algorithms for Data Science}}

In Winter 2024, I taught an undergraduate course on randomized algorithms for data science.\footnote{\url{https://www.rtealwitter.com/rads2024/}}
The framework of the course is an iterative approach for building efficient algorithms from an expanding mathematical toolkit.
We apply the framework to a sequence of data science problems such as how to count unique items in a stream of data, how to search for similar songs using only a few-second audio clip, and how to detect clusters in networks.
Whether they pursue a career in industry or academia, the students can use the framework to iteratively build efficient solutions to challenging data science problems.

During the randomized algorithms course, I developed in-class group activities for students to practice solving problems in a low-stakes environment.
I emphasized it was more important that every group member understood the group solution than that one group member answered the entire problem.
I assigned the groups so that every student practiced both leading group discussions and asking clarifying questions.
I posed problems of varying difficulty so that every student practiced both solving a problem and struggling with one.
The message is that every question and point of confusion is as welcome and normal as every explanation and insight.
In the words of one student:
\begin{displayquote}
	\textit{I really enjoyed the group activities as it allowed me to learn from other people just as much as it was a test of my knowledge of the topic. The notion of working with other people is something that [...] %I feel
	has gone understated in Computer Science, but you learn so much more when you do.}
\end{displayquote}

I craft resources to make mathematical and algorithmic ideas more accessible.
In the semester before I taught the randomized algorithms course, I wrote notes that explained the motivation, techniques, proofs, and possible extensions for every lecture.
I encourage students to read the notes before the class to get a high-level understanding of the topic and after the class to solidify the details of what we covered.
So students could focus on understanding instead of copying what I said in the lecture, I disseminate copies of my handwritten lecture notes.
I offer office hours after every class so that students can get thorough and relevant answers to their questions.
By providing such carefully-crafted resources throughout the class, I support students to learn and practice the material in an accessible environment tailored to their needs.

When asked about my overall teaching effectiveness, one student said:
\begin{displayquote}
	\textit{Teal taught this class remarkably well. He was empathetic and understanding. I always felt like I could approach him with any question without judgement. I could tell that he cared about my learning and encouraged me to learn as much as possible and to think beyond grades. He also handled his own mistakes carefully and was always open to criticism. %It is obvious that Teal is an amazing teacher that cares about his students' learning and experience.
	}
\end{displayquote}

%In addition to my courses, I use the problem-solving framework to guide the student projects I advise.
%The student projects are structured around research questions like:
%``How does Gaussian splatting capture the richness of complex 3D scenes?''
%``Can we guide autoregressive LLMs with a latent `thesis'?''
%``How useful is importance sampling for training neural networks?''
%The projects are based on student interest and are designed to balance learning prior work with exploring new ideas.
%As in my courses, I offer a broad framework for understanding the problem and the students develop the knowledge to apply the framework to the topic at hand.
%Through the advising process, students develop both specific skills relevant to the topic and a broad framework to successfully conduct future research projects.

As a professor, I will bring my framework-based approach and supportive teaching style to future courses, iteratively refining what and how I teach to make every class better than the last. 
In addition to introductory computer science and theoretical courses, I am prepared to teach elective courses on machine learning, quantum computing, and advanced algorithms.
Like my existing courses, I will structure my future classes around a problem-solving framework that transfers to the computer science problems students face beyond the classroom.

\end{document}